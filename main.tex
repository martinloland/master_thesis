\documentclass[b5paper,10pt,twoside]{book}
\usepackage[utf8]{inputenc}
\usepackage{mystyle}

%\input{glossary}
%\input{nomenclature}
%\setlength{\marginparwidth}{2cm}
\pagenumbering{roman}

\begin{document}

\chapter*{\Huge Abstract}
\addcontentsline{toc}{chapter}{Abstract}

\todo[inline]{
\textbf{Todo}

- what you investigated

- how you did it

- what you found out

- novelty
}

\noindent \lipsum[1]
%\clearpage
\chapter*{\Huge Sammendrag}
\addcontentsline{toc}{chapter}{Sammendrag}

\noindent \lipsum[1]
%\clearpage

{\setstretch{1.0}
\tableofcontents
%\clearpage
%\addcontentsline{toc}{chapter}{\listfigurename}
%\listoffigures
}
{\setstretch{1.5}
\pagenumbering{arabic}
\chapter{Introduction}
	\todo[inline]{
	\textbf{Todo}
	
	- a problem or phenomenon you want to study

	- the reasoning behind your choice of topic (gap in knowledge)
	
	- the research question or hypothesis you set out to investigate
	}
	\section{Teleoperation in today's world}
	\section{Engage eduROV project}
	\section{Video latency and user performance}
	\section{Visual cues and predictive displays}
	- Should current work be moved to theory?
	\section{Problem statement}
	
	
\chapter{Theory}
	\section{Video latency}
	\section{Performance degradation}
	\section{Ways to combat latency effects}
	\todo[inline]{
	\textbf{Todo}
	
	- can be divided into how the tools, camera and environment move with respect to each other
	
	- mesh generation
	
	- phantom robot
	
	- Autonomy

	- AR with predicted future 3D geometry
	
	- Having stored local bigger images on cropping that image
	
	- capturing 3d data from scene and predicting the movement in that scene
	
	- displaying a generated 3d scene from remote images
	}
	
\chapter{The eduROV package}
	\section{Development method}
	\todo[inline]{
	\textbf{Todo}
	
	- Git with issues
	
	- Branches
	}
	\section{Features and technology}
	\todo[inline]{
	\textbf{Todo}
	
	- System architecture
	
	- Software structure
	}
	\section{GUI and API design}
	\section{Performance}
	\section{Documentation}
	
\chapter{Methodology}
	\section{Predictive display by real-time transforming video}
		\subsection{Translation calculations}
		\subsection{Software implementation}

	\section{Experimental design}
		\subsection{Unmanned Ground Vehicle}
		\subsection{Task}
		\subsection{Metrics}
		\todo[inline]{
		\textbf{Todo}
		
		- Number of button presses
		
		- change in rate of presses (learning)
		
		- frequency of presses (presses / minute)
		
		- wheel movement
		
		- telepresence
		
		- user satisfaction
		}
	\section{User interaction}
		\subsection{Test group}
		\todo[inline]{
		\textbf{Todo}
		
		- move to results?
		
		- population
		
		- age group
		
		- computer knowledge
		
		- volunteer
		
		- sample size calculations
		}
		\subsection{Questionnaire}
		\todo[inline]{
		\textbf{Todo}
		
		- which questions were asked?
		
		- how where they asked?
		}
		\subsection{Data collection}
	\section{Data interpretation}
	\todo[inline]{
	\textbf{Todo}
	
	- averaged?
	}

\chapter{Results}
	\section{Performance}
	\section{User meanings}
	\section{Statistical analysis}

\chapter{Discussion}
	\todo[inline]{
	\textbf{Todo}
	
	- what kind of display gave the best user performance?
	
	- what did the users have to say about the different displays?
	
	- how does performance and user opinion correlate?
	
	- what does the results mean?
	
	- what are the causes and factors?
	
	- other possible ways to interpret the data?
	
	- what could be done differently?
	}

\chapter{Conclusion}
	\todo[inline]{
	\textbf{Todo}
	
	- to what extent does the presented display help to mitigate the effect of video delay?
	
	- does the findings have any implications on the current state of knowledge?
	
	- what are future research areas?
	}
}

{\setstretch{1.0}
%\printglossary \label{glossary}
%\printnomenclature\label{nomenclature}
}
\printbibliography
\addcontentsline{toc}{chapter}{Bibliography}
\end{document}
