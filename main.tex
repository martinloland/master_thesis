\documentclass[b5paper,10pt,twoside]{book}
\usepackage[utf8]{inputenc}
\usepackage{mystyle}

%\makeglossaries

\newglossaryentry{frame}
{
    name={frame},
    description={a three-dimensional coordinate system which is defined with a position and orientation}
}

\newglossaryentry{inertialframe}
{
    name={inertial reference frame},
    description={reference frame which does not experience any acceleration}
}
 
 
\newglossaryentry{bodyframe}
{
    name={body attached frame},
    description={reference frame for link i with it's origin in the mass center and orientation parallel to the link}
}

\newglossaryentry{DH}
{
    name={Denavit-Hartenberg convention},
    description={convention for describing a link in a serial manipulator robot}
}

\newglossaryentry{homo}
{
    name={homogeneous transformation},
    description={transformation that describes a combination of purely rotation and translation}
}


\newglossaryentry{nef}
{
    name={Newton-Euler method},
    description={A forward-backward recursion method to solve for forces and torques in a linked body system given a position, velocity and acceleration}
}

\newglossaryentry{endeff}
{
    name={end effector},
    description={last part of a chained robot arm, gripper or similar used for interaction with objects}
}

\newglossaryentry{6R}
{
    name={6R robot},
    description={an industrial manipulator with six revolute joints}
}

\newglossaryentry{workspace}
{
    name={workspace},
    description={the working volume of the robot, points where the robot can not reach are outside the workspace of the robot}
}

\newglossaryentry{invkin}
{
    name={inverse kinematic},
    description={calculations that tries to find the joint angles given a wanted position and orientation of the end effector}
}


\newglossaryentry{dirkin}
{
    name={direct kinematic},
    description={calculations that finds the position and orientation of the end effector given a series of joint angles}
}

\newglossaryentry{robot}
{
    name={robot},
    description={in the context of this thesis, an industrial serial manipulator}
}

\newglossaryentry{rov}
{
    name={ROV},
    description={remotely operated vehicle}
}

\newglossaryentry{inforce}
{
    name={inertial forces},
    description={forces and torques required to accelerate an object with mass and inertia}
}
%\usepackage[intoc]{nomencl}
\makenomenclature

\usepackage{etoolbox}
\renewcommand\nomgroup[1]{%
  \item[\bfseries
  \ifstrequal{#1}{L}{Links and joints}{%
  \ifstrequal{#1}{D}{Dynamics}{%
  \ifstrequal{#1}{P}{Physical properties}{}}}%
]}

\nomenclature[L]{$\theta_i$}{link angle for link i in global frame}
\nomenclature[L]{$\omega_i$}{angular velocity for link i in global frame}
\nomenclature[L]{$\alpha_i$}{angular acceleration for link i in global frame}

\nomenclature[L]{$q_i$}{joint angle for link i in body attached frame, \textit{theta} for revolute and \textit{d} for prismatic}
\nomenclature[L]{$\dot{q_i}$}{joint velocity for link i in body attached frame}
\nomenclature[L]{$\ddot{q_i}$}{joint acceleration for link i in body attached frame}

\nomenclature[D]{$\dot{h}_i$}{change in angular momentum for link i}
\nomenclature[D]{$\dot{p}_i$}{change in linear momentum for link i}

\nomenclature[P]{$I_i$}{inertia tensor for link i with respect to it's body attached frame}
\nomenclature[P]{$m_i$}{mass of link i}
%\setlength{\marginparwidth}{2cm}
\pagenumbering{roman}

\begin{document}

\chapter*{Abstract}
\addcontentsline{toc}{chapter}{Abstract}

Teleoperation of remotely operated vehicles has become an increasingly viable solution in many fields as technology has improved and the requirements for risk and cost reduction has increased. When operating vehicles, especially at long distances, unwanted latency is introduced to the system. As a results, cognitive workload increase and performance is degraded. Predictive technology has proven to be an effective method to reduce these effects. But many of the current implementations rely on expensive equipment or extensive knowledge of the robotic system.

A new type of predictive display based on image transformation has been developed. It does not require any additional hardware and can be implemented on a wide range of vehicles without much configuration. This thesis aimed to investigate H1: a simple predictor display based on image transformation can increase the operator performance. And H2: a simple predictor display based on image transformation will decrease the operator's subjective workload.

An experiment was performed where the 58 participants were given a modified "peg-in-hole" task. During a test time of 90 seconds the subjects had to move the vehicle and score as many hits as possible. This was performed using three different conditions. Condition one using a 750ms delay, condition two having a 750ms delay with predictor screen and condition three with a 250ms long delay but no predictive screen.

The results showed that participants performed on average 20.6\% better on condition two with the predictive display versus condition one with no predictive display. The results also showed that particpants who play games weekly or more, got almost twice the benefit from the predictive display. Gamers had a 30.13\% increase while non-gamers only gained a 16.91\% performance increase. The participants reported no statistical difference in their mental, physical and temporal demand. The predictive display did therefore not reduce the subjective workload.

%\clearpage
\chapter*{\Huge Sammendrag}
\addcontentsline{toc}{chapter}{Sammendrag}

\noindent \lipsum[1]
Se \citep{Ricks2004} for mer info.
%\clearpage

{\setstretch{1.0}
\tableofcontents
%\clearpage
%\addcontentsline{toc}{chapter}{\listfigurename}
%\listoffigures
}
{\setstretch{1.5}
\pagenumbering{arabic}
\chapter{Introduction}
	\todo[inline]{
	\textbf{Todo}
	
	- a problem or phenomenon you want to study

	- the reasoning behind your choice of topic (gap in knowledge)
	
	- the research question or hypothesis you set out to investigate
	}
	\section{Teleoperation in today's world}
	\section{Engage eduROV project}
	\section{Video latency and user performance}
	\section{Visual cues and predictive displays}
	- Should current work be moved to theory?
	\section{Problem statement}
	
	
\chapter{Theory}
	\section{Video latency}
	\section{Performance degradation}
	\section{Ways to combat latency effects}
	\todo[inline]{
	\textbf{Todo}
	
	- can be divided into how the tools, camera and environment move with respect to each other
	
	- mesh generation
	
	- phantom robot
	
	- Autonomy

	- AR with predicted future 3D geometry
	
	- Having stored local bigger images on cropping that image
	
	- capturing 3d data from scene and predicting the movement in that scene
	
	- displaying a generated 3d scene from remote images
	}
	
\chapter{The eduROV package}
	\section{Development method}
	\todo[inline]{
	\textbf{Todo}
	
	- Git with issues
	
	- Branches
	}
	\section{Features and technology}
	\todo[inline]{
	\textbf{Todo}
	
	- System architecture
	
	- Software structure
	}
	\section{GUI and API design}
	\section{Performance}
	\section{Documentation}
	
\chapter{Methodology}
	\section{Predictive display by real-time transforming video}
		\subsection{Translation calculations}
		\subsection{Software implementation}

	\section{Experimental design}
		\subsection{Unmanned Ground Vehicle}
		\subsection{Task}
		\subsection{Metrics}
		\todo[inline]{
		\textbf{Todo}
		
		- Number of button presses
		
		- change in rate of presses (learning)
		
		- frequency of presses (presses / minute)
		
		- wheel movement
		
		- telepresence
		
		- user satisfaction
		}
	\section{User interaction}
		\subsection{Test group}
		\todo[inline]{
		\textbf{Todo}
		
		- move to results?
		
		- population
		
		- age group
		
		- computer knowledge
		
		- volunteer
		
		- sample size calculations
		}
		\subsection{Questionnaire}
		\todo[inline]{
		\textbf{Todo}
		
		- which questions were asked?
		
		- how where they asked?
		}
		\subsection{Data collection}
	\section{Data interpretation}
	\todo[inline]{
	\textbf{Todo}
	
	- averaged?
	}

\chapter{Results}
	\section{Performance}
	\section{User meanings}
	\section{Statistical analysis}

\chapter{Discussion}
	\todo[inline]{
	\textbf{Todo}
	
	- what kind of display gave the best user performance?
	
	- what did the users have to say about the different displays?
	
	- how does performance and user opinion correlate?
	
	- what does the results mean?
	
	- what are the causes and factors?
	
	- other possible ways to interpret the data?
	
	- what could be done differently?
	}

\chapter{Conclusion}
	\todo[inline]{
	\textbf{Todo}
	
	- to what extent does the presented display help to mitigate the effect of video delay?
	
	- does the findings have any implications on the current state of knowledge?
	
	- what are future research areas?
	}
}

{\setstretch{1.0}
%\printglossary \label{glossary}
%\printnomenclature\label{nomenclature}
}
\printbibliography
\addcontentsline{toc}{chapter}{Bibliography}
\end{document}
