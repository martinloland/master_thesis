\chapter*{Sammendrag}
\addcontentsline{toc}{chapter}{Sammendrag}

Fjernstyring av roboter har blitt et stadig mer populært alternativ til tradisjonelle operasjoner etter hvert som teknologien har blitt tilgjengelig og kravene til sikkerhet og økonomistyring har økt. Når roboter fjernstyres, spesielt fra lange distanser, oppstår det uønsket tidsforsinkelse i systemet. Som et resultat øker den kognitive påkjennelsen og operasjonseffektiviteten synker. Prediktiv teknologi har vist seg å være et bra alternativ for å minske de negative effektene. Mange av de nåværende løsningene har dog krav til avansert utstyr eller omfattende informasjon om roboten.

En ny type prediktivt grensesnitt basert på forskyvning og skalering av video har blitt utviklet. Dette grensesnittet krever ikke ekstra utstyr og kan anvendes på en rekke forskjellige robotkonfigurasjoner. Denne masteroppgaven ønsket å undersøke følgende påstander. H1: et prediktivt grensesnitt basert på forskyvning og skalering av video kan øke operasjonseffektiviteten og H2: et prediktivt grensesnitt basert på forskyvning og skalering av video vil senke den subjektive kognitive påkjennelsen.

Et eksperiment ble utført hvor 58 deltakere ble gitt en oppgave hvor de måtte styre en robot inn i en rekke hull i løpet av 90 sekunder. Denne testen ble gjennomført under tre forskjellige betingelser. Første inneholdt en tidsforsinkelse på 750ms, den andre inneholdt den samme tidsforsinkelsen, men med det prediktive grensesnittet. Den siste betingelsen hadde en forsinkelse på 250ms og ingen ekstra hjelp.

Resultatene viste at deltakerne utførte oppgaven 20.6\% bedre under betingelsen som inneholdt det prediktive grensesnittet kontra den samme tidsforsinkelsen uten prediktivt grensesnitt. Resultatene viste også at personer som spiller videospill på en ukentlig basis gjorde det bedre enn resten. De hadde en positiv økning på 30.13\%, mens resten av deltakerne oppnådde 16.91\%. Deltakerne rapporterte ingen statistisk forskjell når det kom til fysisk, mental eller stressende påkjenning. Det prediktive grensesnittet hadde derfor ingen reduksjon på den subjektive kognitive påkjennelsen.
