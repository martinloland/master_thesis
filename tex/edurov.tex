\section{Python}

\todo[inline]{
- popularity and users

- Python Package Index PyPi
}

\section{Current alternatives}

\todo[inline]{
- ROS

- GoPiGo

- creating your own
}

in the situations where you are creating your own there are some issues: require apps, exposed to all the code, hard to change, combination of multiple packages, low support for parllellism, don't support keyboard use, insetad buttons, no documentation

\section{Development}

\todo[inline]{
- Git with issues

- Branches

- PyPi and versioning scheme
}

\section{Architecture}
\todo[inline]{
- socket protocols / Pyro4

- HTTP Webserver, ethernet and wifi

- different processes / Parallelism

- camera capture
}

\missingfigure{Graphics showing the architecture, hardware and software}

\section{Graphical user interface}
\todo[inline]{
- html and js

- customizable

- in the context of engage

- attitude
}

\citep{Chen2007} "Attitude (i.e., pitch and roll) of a robotic vehicle may be easy to reference when there are other familiar objects (e.g., horizon, buildings, trees, etc.) in the re- mote environment.However, if those reference points are absent and the on-board cameras are fixed, operators sometimes find it surprisingly hard to accurately assess the attitude of their robotic vehicles."

\citep{Wang2004} Gravity-Referenced Attitude Display for Teleoperation of Mobile Robots

\citep{Chen2007} "The ideal view depends on the task; overall awareness and pattern recognition are optimized by exocen- tric views whereas the immediate environment is often viewed better egocentrically."
"overlaying information on video feed can potentially lead to cognitive tunneling"

\missingfigure{Image showing the GUI of the eduROV submersible}

\section{Application programming interface}

\todo[inline]{
- WebMethod
}

\section{Performance}

\missingfigure{Image showing the video latencies at different resolutions}

\section{Documentation}

\todo[inline]{
- Read the docs with git

- Restructured text

- Autodoc with git

- Examples
}

\missingfigure{example code of how the documentation is written}

\missingfigure{example image of how this text is shown in the web browser}

\section{Novelty features}

\todo[inline]{
- main features from docs

- what does it do better than the competition
}