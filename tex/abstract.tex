\chapter*{Abstract}
\addcontentsline{toc}{chapter}{Abstract}

Teleoperation of remotely operated vehicles has become an increasingly viable solution in many fields as technology has improved and the requirements for risk and cost reduction has increased. When operating vehicles, especially at long distances, unwanted latency is introduced to the system. As a results, cognitive workload increase and performance is degraded. Predictive technology has proven to be an effective method to reduce these effects. But many of the current implementations rely on expensive equipment or extensive knowledge of the robotic system.

A new type of predictive display based on image transformation has been developed. It does not require any additional hardware and can be implemented on a wide range of vehicles without much configuration. This thesis aimed to investigate H1: a simple predictor display based on image transformation can increase the operator performance. And H2: a simple predictor display based on image transformation will decrease the operator's subjective workload.

An experiment was performed where the 58 participants were given a modified "peg-in-hole" task. During a test time of 90 seconds the subjects had to move the vehicle and score as many hits as possible. This was performed using three different conditions. Condition one using a 750ms delay, condition two having a 750ms delay with predictor screen and condition three with a 250ms long delay but no predictive screen.

The results showed that participants performed on average 20.6\% better on condition two with the predictive display versus condition one with no predictive display. The results also showed that particpants who play games weekly or more, got almost twice the benefit from the predictive display. Gamers had a 30.13\% increase while non-gamers only gained a 16.91\% performance increase. The participants reported no statistical difference in their mental, physical and temporal demand. The predictive display did therefore not reduce the subjective workload.
