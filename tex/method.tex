predictor display in own category
 
Predictive Display Model

\citep{Zheng2016}:
 "Besides robustness, a model-free approach would also have the benefit of decoupling the predictor design from the vehicle design, thus providing more flexibility for using the same predictor for different vehicles. Furthermore,"

method:
- intro
- participants
- equipment / apparatus
- experimental design / task
- procedure
- data recording

\citep{Lu2018} "Condition order followed a 3x3 Latin-Square Design, so as to eliminate the order effect"

"(Three) metrics are used to quantify performance;"

"steering control effort is monitored to assess drivability" (TLX)

\section{Predictive display by real-time transforming video}
\todo[inline]{
\textbf{Todo}

- very similar to \citep{Matheson2013} projected display, does not seem to handle rotation very well, does include an arrow as a avatar robot

- it is normal to generate a 3d environment, but is that really necessary?

- is it enough to show the movement by translating and scaling the video?
}
    \subsection{Translation calculations}
    
    \citep{Matheson2013} "Also, since the predictive displays are merely modifications of the most recent video, any errors due to modelling and prediction are not cumulative."
    
    \subsection{Software implementation}
    \todo[inline]{
    \textbf{Todo}
    
    -javascript delay function
    }

\section{Experimental design}
	\subsection{Teleoperation of robot in a room}
	\missingfigure{Image of robot}
	\todo[inline]{
	\textbf{Todo}
	
	- "human subject experiments"
	
	- keep it simple to reduce variance i results, but complex enough that the effects of video delay would be noticeable (and predictor could help)
	
	- avoid ceiling effects (to slow robot/to easy tasks)
	
	- running a warmup where the user familiarizes himself with the input device and kinematics of the robot
	
	- random order for all displays, time for warmup
	
	-where is the user looking
	
	- "delay conditions were presented to the participants in a counterbalanced order to control for possible learning effects."
	}
    \subsection{Unmanned Ground Vehicle}
    \todo[inline]{
    \textbf{Todo}
    
    - how mas the rc car made
    
    - circuit pihat design
    
    - software
    
    - 2 DOF, rotation and forward/backward
    }
    \subsubsection{Video latency}
    \todo[inline]{
    \textbf{Todo}
    
    - video delay measurement
    
    - relevant delay for wireless high definition low cost solution, or very long range (digital)
    
    - "apparent response delay"
    
    - Natural delay
    
    - Added delay
    }
    \subsection{Information}
    \todo[inline]{
    \textbf{Todo}
    
    - the goal of the experiment
    
    - be aware of getting to close to the wall
    
    - rocker wheel at the back
    
    - information page of the task
    
    - were NOT told that one of the displays is a predictor displays
    }
    \subsection{Task}
    \subsection{Metrics}
    \todo[inline]{
    \textbf{Todo}
    
    - focus on speed, not so much on accuracy / collisions
    
    - Number of button presses
    
    - change in rate of presses (learning)
    
    - frequency of presses (presses / minute)
    
    - wheel movement
    
    - telepresence
    
    - user satisfaction
    }
\section{User interaction}
    \subsection{Test group}
    \todo[inline]{
    \textbf{Todo}
    
    - move to results?
    
    - population
    
    - age group
    
    - computer knowledge
    
    - different crowds
    
    - volunteer
    
    - sample size calculations
    }
    \subsection{Questionnaire}
    \todo[inline]{
    \textbf{Todo}
    
    - how demanding each task was from the operator's points of view.
    
    - nasa tlx load index with likert scale
    
    - which questions were asked?
    
    - how where they asked?
    }
    \subsection{Data collection}
    \todo[inline]{
    \textbf{Todo}
    
    - html forms with sqlite3 database
    }
\section{Data interpretation}
\todo[inline]{
\textbf{Todo}

- t-test and effect size
}