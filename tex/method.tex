\todo[inline]{
- experiment was conducted with the goal of..

- three different displays

- task of controlling a small UGV to push buttons with and without predictor display and added delay
}

\section{Participants}
\todo[inline]{
- Numbers

- Demographics, age, computer knowledge
}

\section{Experimental design}
\todo[inline]{
- robot

- controlling (can not see robot)

- delay

- task

- testing Fitt's law

- layout of the button case
}

keep it simple to reduce variance i results, but complex enough that the effects of video delay would be noticeable (and predictor could help)

"human subject experiments"

avoid ceiling effects (to slow robot/to easy tasks)

robot software, electronics, equipment, 2 DOF, rotation and forward/backward

relevant delay for wireless high definition low cost solution, or very long range (digital), what are the delays


\section{Procedure}
\todo[inline]{
- groups

- what information were they given

- training

- time of experiment

- reposition of robot

- demographics at start

- answered survey for each exp
}

running a warmup where the user familiarizes himself with the input device and kinematics of the robot

\citep{Lu2018} "Condition order followed a 3x3 Latin-Square Design, so as to eliminate the order effect"

"delay conditions were presented to the participants in a counterbalanced order to control for possible learning effects."

info: goal of exp, be awera of wall, rocker wheel, information page

were NOT told that one of the displays is a predictor displays


\section{Data recording}
\todo[inline]{
- Hits / Score at time

- camera

- NASA TLX

- keypresses

- surveys

- SQLite
}

"(Three) metrics are used to quantify performance;"

"steering control effort is monitored to assess drivability" (TLX)


\section{Performance analysis}
\todo[inline]{
- how was the performance quantified

- normalization

- t-test when comparing two displays, ANOVA when comparing three

- t-test: scipy.stats.ttest\_rel

- cohens d: appendix

- \citep{Rachmielowski2010} and \citep{Lovi2014} Normalized data the same way

- APA style

- effect size

- paried sample t-test

- T value means that they are 4 times more different from each other than they are within
}
