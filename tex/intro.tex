The structure of this report is as follows: in this chapter, the applications and challenges with teleoperation will be discussed. In addition, a walk trough of the most popular methods in predictive technology to combat time delay issues will be presented.

Chapter \ref{chpEdurov} is dedicated to the development of the python package named \textit{eduROV} that was created in conjunction with the Engage eduROV summer school project at Norwegian University of Science and Technology (NTNU). The developed software were also used in the main experiment of this report. For those only concerned in the predictive technology and relevant research, this chapter can be skipped.

Next, in chapter \ref{chpPredictive} the developed predictor display based on translation movement will be presented. It will explain how the movement of a robot is estimated and how predicted changes are visualized.

Lastly, in chapter \ref{chpMethod} through \ref{chpDiscussion} the method, results and discussion is presented. Only small sections of the code and recorded data is presented in the chapters, but the rest can be found in the appendix at page \pageref{appendix}. \textbf{TODO}

\section{Teleoperation}


\subsection{Applications}
Teleoperation of vehicles has gained popularity since it first became possible. This includes underwater, ground, aerial and space vehicles. The controlled vehcicle is referred to as a \emph{remotely operated vehicle} (ROV). There are many locations and tasks where ROV's are useful. These includes places that are to risky for people, like post disaster areas, underwater operations, space, conflict zones etc. It can also be that using humans is to costly or just impossible. Offshore maintenance and heavy duty mining are some of the tasks. The word \emph{robot} will be used interchangeably with ROV throughout this thesis.

In this report the focus has been on teleoperation by the means of camera feed from the ROV (slave), to the remotely human operator (master) and then control signals from operator to the ROV. This form of teleoperation can be effective because it's easy to implement and understand for the operator. Other forms of teleoperation can be without camera feed or by increasing the level of autonomy (LOA). This will not be any focus in this report.

In addition, only unmanned vehicles will be considered. Although a unmanned ground vehicle (UGV) has been used in the experiment, the findings can be applied to teleoperation of all types of vehicles, that be aerial, ground, underwater and space. It does however not apply to situations where the camera is in a fixed position and rotation. This configuration is often used in telemedicine \citep{Kumcu2017} or robot arm manipulation \citep{Bejczy1990}.

\subsection{Telepresence}

\citet{Draper1998} defined telepresence as "the perception of presence within a physically remote or simulated site". He also states that "telepresence is generally hypothesized to improve efficiency or reduce user workload" and that telepresence is beneficial to mission performance.

\citet{Chen2007} went through 150 papers and checked different teleoperation factors and how they influence user performance. They found eight main factors; Field of view (FOV), orientation, camera viewpoint, depth perception, video quality and frame rate (FR), time delay, and motion. FOV describes the amount of environment that is visible in the video. Orientation (in the environment) can be difficult to perceive if there is mismatch or lack of information. Camera viewpoint is often \emph{egocentric} (robot view) or \emph{exocentric} (birds view) which can lead to tunneling or loss of true ground view respectively. Lack of depth perception can cause wrong estimation of distances and video quality can reduce target identification. Time delays effects are very task dependent but often cause reduced driving performance. Motion describes the situation where the operator itself is moving and can cause motion sickness.


\subsection{Time delay}
Among the factors mentioned above, time delay or \emph{latency} has been found to have big impacts one teleoperation performance \citep{Chen2007}. Chen also noted that latencies as low as $10-20$ ms can be detected by people. \citep{Arthur1993} found latencies (ranging from 50 to 550 ms) to be a more important factor than FR (30, 15, or 10 fps). 

Time delay introduces a situation where the operator's commands does not correspond to the visual feedback he or she is getting. Because of this human drivers tend to over steer and oscillate with their \emph{correcting} steering commands \citep{Appelqvist2007}. This increase the cognitive workload because the operator has to remember the input already given when giving new control commands \citep{Matheson2013}. And \citep{Ricks2004} found that the mental load required to keep track of the robot pose adversely affects the operator's ability to effectively control the robot. A principle of reducing the workload is therefore to maintain correlation between commands issued by operator and changes in the interface \citep{Nielsen2007}. 

A summary of some reports can be seen in table \ref{reviewPerf}. It shows the increase factor for different tasks and delay times. An increase factor of 1.40 is equal to a 40\% increase in task completion time. The actual detrimental effect of latency is very task dependent. In the table, the same increase factor of 1.5 can be found at 100ms for a needle-driving task, and in the robot car movement task at 2000ms.
Some argue that task completion time increase linearly with delay time \citep{Ando1999}, \citep{Lane2002}. While others experience an exponential increase \citep{Xu2014}.

\begin{table}[b]
\centering
\caption{Task completion time increase factor for different delays. N = number of participants}
\label{reviewPerf}
\resizebox{\columnwidth}{!}{
\begin{tabular}{@{}lllccc@{}}
\toprule
Author        & Task               & N  & \multicolumn{3}{c}{Time delay {[}ms{]} and increase factor} \\ \midrule
              &                    &    & 100-300       & 400-700       & 800-1500     \\ \cmidrule(l){4-6} 
\cite{Fabrizio2000}  & Pin transfer       & 6  & 1.04-1.21*     & 1.17-1.41*     & 1.11-1.58*     \\
\cite{Xu2014}        & Energy dissection  & 16 & 1.4-1.8       & 2.7-4.3       &              \\
\cite{Xu2014}        & Needle-driving     & 16 & 1.5-2.1       & 2.5-6.2       &              \\
\cite{Perez2016}     & Surgical simulator & 37 & 0.75          & 1.5           &              \\
\cite{Lum2009}       & Block transfer     & 14 & 1.45          & 2.04          &              \\
\cite{MacKenzie1993} & Target acquisition & 8  & 1.64          &               &              \\ \bottomrule
\multicolumn{6}{l}{* Estimated from graph}
\end{tabular}
}
\end{table}

The reasons for time delay can be many and is not the focus in this report. An overview can be seen in figure \textbf{TODO}. Some of the factors are distance between robot and operator, equipment technology, transfer method, interference, video compression, control algorithms efficiency and many more.

\missingfigure{Graphics showing the different stages in communication that contribute to the latency}


\subsection{Delay compensation}

There are three main ways to combat the detrimental effects of time delay. First, by increasing the level of automation (LOA), the operator workload is reduced. The results of \citep{Luck2006} showed that the higher LOA, the better performance in terms of both time and number of errors made. In some cases, such as a communication blackout, autonomy is essential \citep{Dorais1999}. This option is not always available and may not even be possible. It could require very advanced hardware and software, depending on the task. \citep{Goodrich2001} argued that adjustable autonomy could be used to increase the robot effectiveness. He also mentions that a more autonomous robot is required when longer time delays are present. On the other hand, he also mentions that \emph{"as autonomy level increases, the breadth of tasks that can be handled by a robot decreases."}

Secondly, instead of increasing LOA, providing more information to the operator may increase situational awareness and therefore performance. \citep{Miller2005} performed an experiment where the operator were reminded of what commands had been given by \emph{"providing them with a streaming command indicator."} The preliminary results showed that the operator reported lower fatigue levels. But there are limitations to how much information an operator can digest in a finite amount of time. \citep{Chen2007} explained that overlaying information on video feed can potentially lead to cognitive tunneling.

Lastly, as a third option, there is the use of \emph{predictive technology}. These are displays, control algorithms and graphical models that try to predict the future state of the ROV. They are based on the vehicle's current state and the operator commands that has been given. \citep{Chen2007} which reviewed multiple experiments about time delay performance, conclude that if \emph{"these delays cannot be engineered out of the system, it is suggested that predictive displays or other decision support be provided to the operator".}


\section{Predictive technology}

\begin{wraptable}{r}{0.4\textwidth}
\sbox0{
\resizebox{0.8\paperheight}{!}{
\begin{tabular}{@{}llllllll@{}}
\toprule
Author                   & Robot system              & Task                       & Predictor method                         & Camera              & Participants & Delay          & Task time reduction  \\ \midrule
\cite{Lu2018}            & Car simulator             & Driving                    & Model-free framework                     & Simulated human     & 12           & Not reported   & 8\%                  \\
\cite{Zhang2017}         & 9DOF dynamics simulator   & Rendezvous and docking     & Dynamic prediction from state equations  & Fixed to spacecraft & 9            & 6000           & Not reported         \\
\cite{Arai2016}          & 6DOF robot arm            & Pick and place             & Robot postion calculation                & Virtual             & 3            & Not reported   & Not reported         \\
\cite{Hu2016}            & 2-6 DOF manipulator       & Camera aligment            & Simulated 3D scene                       & Virtual             & 15           & 300, 500, 1000 & 33\%, 58\%, 65\%     \\
\cite{Zheng2016}         & Car simulator             & Driving                    & Model-free framework                     & Simulated human     & 5            & 900            & 35\%                 \\
\cite{Hu2015}            & 4 wheeled car             & Manouvering                & Image tracking and geometry construction & Fixed to car        & No tests     &                & Not reported         \\
\cite{Lovi2010}          & Robot arm on Segway       & Object alignment           & Vision-based monocular modeling          & At end effector     & 5            & 300            & 33\%                 \\
\cite{Matheson2013}      & Rover                     & Driving                    & Projected field of view estimation       & Fixed to car        & 12           & 3000           & 48\%-64\%            \\
\cite{Rachmielowski2010} & Virtual with Phantom OMNI & Alignment                  & Reconstructed 3D environment             & At end effector     & 12           & 300            & 29\%-30\%            \\
\cite{Baldwin1999}       & Stationary camera         & Identify and select shapes & Cropped panoramic video                  & Panoramic 360       & No tests     &                & Not reported         \\
\cite{Mathan1996}        & Lunar vehicle             & Manovuering                & Superimposed directional information     & Fixed to car        & 8            & 5000           & 24\%-30\%            \\
\cite{Bejczy1990}        & 6DOF PUMA robot           & Tapping                    & Superimposed phantom robot               & Fixed               & 2            & 1000, 4000     & 13\%-34\%, 40\%-56\% \\ \bottomrule
\end{tabular}
}
}
\rotatebox{-90}{%
\parbox{\wd0}{\centering
\caption{Predictive technology experiments with task time reduction. Ordered by date. Some values estimated from graphs.}
\label{revirePred}
\endgraf
\usebox0}}
\end{wraptable}

Table \ref{reviewPred} shows a summary of some experiments that has been done in the field of predictive technology. Some of the papers do not report numerical values, the \emph{task time reduction} is then reported as an estimation from graphs. The experiments spans a wide variety of robot configurations, experiment tasks and predictive methods and can not necessary be compared directly.


The robot system can be roughly divided into two main groups; either the robot configuration can be exactly calculated in real time. This includes robot arm manipulators fixed to a defined reference frame where it's configuration is a result of user input only. Or secondly, the robot configuration is subjected to external forces or freely floating. ROV's typically goes in this group since they are able to move around in the environment. This makes the prediction more complicated because unknown and changing external factors has to be considered.

There is also great variety in the tasks used in table \ref{reviewPred}. They do however have one thing in common; they all include some sort of lateral movement. Typically the operator is requires to perform an alignment or aiming task. These kinds of tasks are particular exposed to the detrimental effects from communication delay. It can cause the operator to \emph{overshoot} the target and transition to an inefficient \emph{move and wait} strategy which can be measured by task completion time. \citep{Lane2002} noted that this behavior started to happen at around one second of time delay.

In all kinds of predictive technology a future predicted state of the robot has to be calculated. Which variables that are used for this and how it's calculated varies. Some methods rely on the dynamic equations of the system. \citep{Zhang2017} implemented a version where he used the state equations of a spacecraft and it's dynamic properties to calculate it's predicted state. The operator where then presented with a future predicted image of the spacecraft and gave commands correspondingly. This can be a good approach in space since all external forces can be accurately modeled.

In situations where the external forces can not be calculated exactly and the ROV is free to move around, a \emph{model free} approach (no dynamics) are often used instead. The method of conveying this information can be divided into three groups and are presented below.


%\citep{Chen2007} "Woods et al. [16] suggested that achieving functional presence might be a more realistic goal for teleoperation user interface design. Functional presence occurs when the teleoperator receives sufficient perceptual cues to effectively conduct teleoperations"


\subsection{Superimposed predictive information}

%\citep{Mathan1996} "predictive display superimposing directional velocity information on the video display"

In this category, predictive information is overlayed or \emph{superimposed} on the delayed video feed. In that way the operator is able to see estimates on where the ROV is going to end up. The prediction is often visualized as vector graphics in the form of lines or points along a path. \citep{Mathan1996} used this approach when he superimposed directional velocity information about a lunar rover on a video display.

A similar example can be seen in airplanes and helicopters where a \emph{tunnel in the sky} display shows where the aircraft should be going and a cross indicating the predicted trajectory \citep{Grunwald1981}. In cases with big amounts of lateral movement this approach might not be applicable because the predicted heading can come off screen.

\subsection{3D graphic models}

About half of the experiments in table \ref{reviewPred} would adhere to this category. Generally, a 3D world is constructed from sensory input such as laser ranging, stereo cameras, image tracking or others. Images taken by normal cameras is then mapped to the surface of the computer generated world. Lastly, a virtual camera is placed inside the virtual world in the predicted position of the real camera. The operator is then given the virtual video feed as virtual reality (VR), or in a combination with the real one, augmented reality (AR). As \citep{Hu2016} describes it; \textit{"In [a] VR-based Predictive display (PD), instead of delayed visual feedback from the remote robot site, an immediately and predicted visual feedback is rendered from a graphics model in response to the operator's motion command"}. \citep{Kim1993} is one example of this.

Some of the technologies used for capturing the 3D world are \textit{Monocular Simultaneous Location and Mapping} (SLAM), stereo imagery, vision-based structure from motion (SFM), light detection and ranging (LiDAR) or radio detection and ranging (radar).

This method is particular popular in conjunction with robot arm manipulators. In these cases the 3D environment can be constructed in advance and the exact location of the robot arm is known, \citep{Ricks2004}. A limitation with this approach comes in areas where the tasks are performed unknown and unstructured areas. Then geometry can not be created in advance and real time mapping and rendering can be difficult. In addition, it can require additional hardware such as stereo cameras and the calculations can be computer intensive.

%\citep{Burkert2004} "Those predictions are only possible if models of the geometry, kinematics, and dynamics of the remote scene are locally available."


\subsection{Video manipulation}

Instead of constructing a virtual 3D world a more simple approach is possible. This approach tries to make alterations to the delayed video such that it looks like the ROV is actually moving in real time. A simple example would be to zoom into the image if the robot is moving forward. \citep{Matheson2013} halved the task completion time at a latency of three seconds in his experiment. He used an approach where the \textit{"display is obtained by estimating the current rover position within the delayed drive camera image, finding the current field of view edges given the rover’s location and orientation, and manipulating the delayed image through cropping and projection, to approximate the view from the current rover location".}

A similar method is by capturing a wide FOV video, possibly 360 degrees, and then only displaying a section of that image to the operator. The section can then be moved around in the video as a response the operator's commands and thus provide fluid and seemingly real time feedback, \citep{Baldwin1999}.

The approach of video manipulation has the advantages of being low cost, easy to implement, does not rely on a structured environment and since the displayed video are merely alterations to the last image, no prediction error propagation will happen. It is however not able to recreate \textit{parallax} movement, which can be achieved using the 3D model method above. An example of parallax movement would be when passing a corner or object. New parts of the environment should be visible, but can not be constructed from a delayed image.

\section{Problem statement}

Most of the previous research seems to be concerned about 3D environment reconstruction from sensory data. While it shows promising results and great reduction in task completion times, this is a method that requires advanced algorithms and possibly expensive equipment. Many of the mentioned predictive technologies also require extensive information about the environment and the robot in order to function. This is either not possible or a time consuming task. In this context, the video manipulation method provides and easy and cheap way to increase operator performance.

The \textit{projected display} method described by \citep{Matheson2013} is probably the easiest video manipulation method while still providing good performance increase. The experiment simulated a lunar rover with a latency of three seconds and had twelve participants.

There seems to be a lack of quantitative research on the topic if simple video manipulation predictor displays, used to control ROV's with a medium communication delay in the range of 500-3000 ms. Therefore, this thesis aimed to study the following hypotheses:

\begin{itemize}
\item H1: a simple predictor display based on image transformation can 
 increase the operator performance
 \item H2: a simple predictor display based on image transformation will decrease the operator's subjective workload
\end{itemize}

\section{Engage eduROV}

This master thesis was conducted in conjunction with the Engage eduROV project. An image of the ROV used in this project can be seen in figure \textbf{TODO}. A software for handling video transmission and robot communication for this ROV was developed as part of the project. This software was also used as a framework for controlling the ROV used in the experiment. In addition, the software gathered quantitative data during the experiment. For more information about the software and it's development, see chapter \ref{chpEdurov}.

\missingfigure{Image of the eduROV submersible}
