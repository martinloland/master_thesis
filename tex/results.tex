\section{Performance}

\figref{performanceNorm} shows the normalized score, number of hits in 90 seconds, for each of the display types and all participants. A similar figure with non-normalized values can be found in the appendix at page \textbf{TODO}. \textit{Delay} refers to the \textit{added} delay, which means that "No delay" translates to the inherent system delay of $250 ms$. The numerical score values together with the standard deviation is reported in table \ref{score}. In addition, the percentage difference in means between displays is also reported in that table. The statistical significance and effect size between conditions can be seen in table \ref{score2}.

Using normalized scores, participants performed on average 20.29\% better using the predictive display versus no predictive display, $t(56)=4.82$, $p<.001$, $d=0.735$. In comparison, the subjects performed on average 153.32\% better in the no delay condition versus the delayed condition without predictive help, $t(56)=23.04$, $p<.001$, $d=4.413$. All differences are statistically significant.


\begin{figure}[h!]
    \centering
    \includegraphics[scale=0.85]{performance_norm}
    \caption{Normalized score for each display type}
    \label{performanceNorm}
\end{figure}

% Please add the following required packages to your document preamble:
% \usepackage{booktabs}
\begin{table}[]
\centering
\caption{Normalized mean$\pm$SD of display scores and mean differences.}
\label{score}
\begin{tabularx}{\textwidth}{@{}lYYYY@{}}
\toprule
Display  & Score                       & \multicolumn{3}{c}{Mean difference from} \\ \cmidrule(l){3-5} 
         &                             & Delay        & Delay PD    & No delay    \\ \midrule
Delay    & 6.24 $\pm$ 1.39 & -            & -17.10\%      & -60.69\%    \\
Delay PD & 7.52 $\pm$ 1.43 & 20.62\%      & -           & -52.58\%    \\
No delay & 15.87 $\pm$ 1.99 & 154.37\%     & 110.88\%    & -           \\ \bottomrule
\end{tabularx}
\end{table}
% Please add the following required packages to your document preamble:
% \usepackage{booktabs}
\begin{table}[]
\small
\centering
\caption{Mean difference, paired samples t-test and Cohen's d effect size for display pair scores. Gamers = plays weekly or more.}
\label{score2}
\begin{tabularx}{\textwidth}{@{}lllYYYY@{}}
\toprule
\multicolumn{2}{c}{Group / Pair}    & Mean difference & \multicolumn{3}{c}{t-test for Equality of Means} & d     \\ \midrule
\multicolumn{2}{l}{}                &                 & t               & df          & p                &       \\ \cmidrule(lr){4-6}
\multicolumn{2}{l}{All N=57}        &                 &                 &             &                  &       \\
Delay             & Delay PD        & 20.62\%         & 4.80            & 56          & $<$.001          & 0.904 \\
Delay             & No delay        & 154.37\%        & 23.15           & 56          & $<$.001          & 5.569 \\
Delay PD          & No delay        & 110.88\%        & 19.66           & 56          & $<$.001          & 4.772 \\\addlinespace
\multicolumn{2}{l}{Gamers n=17}     &                 &                 &             &                  &       \\
Delay             & Delay PD        & 30.13\%         & 4.34            & 16          & $<$.001          & 1.376 \\
Delay             & No delay        & 174.64\%        & 14.93           & 16          & $<$.001          & 6.463 \\
Delay PD          & No delay        & 111.05\%        & 10.83           & 16          & $<$.001          & 4.965 \\\addlinespace
\multicolumn{2}{l}{Non-gamers n=40} &                 &                 &             &                  &       \\
Delay             & Delay PD        & 16.91\%         & 3.20            & 39          & .003          & 0.731 \\
Delay             & No delay        & 146.46\%        & 18.16           & 39          & $<$.001          & 5.237 \\
Delay PD          & No delay        & 110.80\%        & 16.21           & 39          & $<$.001          & 4.655 \\ \bottomrule
\end{tabularx}
\end{table}


\section{Task load index}
\todo[inline]{
- NASA TLX
}

\begin{figure}[h!]
    \centering
    \includegraphics[scale=0.85]{nasa_tlx_bar}
    \caption{NASA TLX (task load index) results for each display type}
\end{figure}