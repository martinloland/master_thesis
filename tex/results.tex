\begin{description}
\item[H1:] Participants performed on average 20.62\% better using the predictive display versus no predictive display, t(56)=4.80, p$<$.001, d=0.904. H1, that a simple predictor display based on image transformation can increase the operator performance, is therefore verified.

\item[H2:] The participants did not reported any significant difference in the mental, physical or temporal demand using the predictive display. H2, that a simple predictor display based on image transformation will decrease the operator's subjective workload, has to be rejected. 
\end{description}
\vspace{-5mm}
\section{Performance}

\figref{performanceNorm} shows the normalized number of hits in 90 seconds (score), for each display type and all N=57 participants. \textit{Delay} refers to the \textit{added} delay, which means that "No delay" translates to the inherent system delay of $250 ms$. The numerical values are reported in Table \ref{score}. The statistical significance and effect size between conditions can be seen in Table \ref{score2}.

\begin{figure}[h!]
    \centering
    \includegraphics[scale=0.85]{performance_norm}
    \caption{Normalized score all participants, N=57.}
    \label{performanceNorm}
	\vspace{-0.2cm}
\end{figure}

All  Those who play games weekly or more were defined as \emph{gamers}. They performed on average 30.13\% better, while non-gamers only saw a 16.91\% performance increase with the PD. This difference is illustrated in \figref{gamer_performance}.

% Please add the following required packages to your document preamble:
% \usepackage{booktabs}
\begin{table}[]
\centering
\caption{Normalized mean$\pm$SD of display scores and mean differences.}
\label{score}
\begin{tabularx}{\textwidth}{@{}lYYYY@{}}
\toprule
Display  & Score                       & \multicolumn{3}{c}{Mean difference from} \\ \cmidrule(l){3-5} 
         &                             & Delay        & Delay PD    & No delay    \\ \midrule
Delay    & 6.24 $\pm$ 1.39 & -            & -17.10\%      & -60.69\%    \\
Delay PD & 7.52 $\pm$ 1.43 & 20.62\%      & -           & -52.58\%    \\
No delay & 15.87 $\pm$ 1.99 & 154.37\%     & 110.88\%    & -           \\ \bottomrule
\end{tabularx}
\end{table}
% Please add the following required packages to your document preamble:
% \usepackage{booktabs}
\begin{table}[]
\small
\centering
\caption{Mean difference, paired samples t-test and Cohen's d effect size for display pair scores. Gamers = plays weekly or more.}
\label{score2}
\begin{tabularx}{\textwidth}{@{}lllYYYY@{}}
\toprule
\multicolumn{2}{c}{Group / Pair}    & Mean difference & \multicolumn{3}{c}{t-test for Equality of Means} & d     \\ \midrule
\multicolumn{2}{l}{}                &                 & t               & df          & p                &       \\ \cmidrule(lr){4-6}
\multicolumn{2}{l}{All N=57}        &                 &                 &             &                  &       \\
Delay             & Delay PD        & 20.62\%         & 4.80            & 56          & $<$.001          & 0.904 \\
Delay             & No delay        & 154.37\%        & 23.15           & 56          & $<$.001          & 5.569 \\
Delay PD          & No delay        & 110.88\%        & 19.66           & 56          & $<$.001          & 4.772 \\\addlinespace
\multicolumn{2}{l}{Gamers n=17}     &                 &                 &             &                  &       \\
Delay             & Delay PD        & 30.13\%         & 4.34            & 16          & $<$.001          & 1.376 \\
Delay             & No delay        & 174.64\%        & 14.93           & 16          & $<$.001          & 6.463 \\
Delay PD          & No delay        & 111.05\%        & 10.83           & 16          & $<$.001          & 4.965 \\\addlinespace
\multicolumn{2}{l}{Non-gamers n=40} &                 &                 &             &                  &       \\
Delay             & Delay PD        & 16.91\%         & 3.20            & 39          & .003          & 0.731 \\
Delay             & No delay        & 146.46\%        & 18.16           & 39          & $<$.001          & 5.237 \\
Delay PD          & No delay        & 110.80\%        & 16.21           & 39          & $<$.001          & 4.655 \\ \bottomrule
\end{tabularx}
\end{table}

\begin{figure}[h!]
    \centering
    \includegraphics[scale=0.85]{gamer_performance}
    \caption{Performance of gamers versus non-gamers}
    \label{gamer_performance}
\end{figure}

\clearpage
\section{Task load index}

\figref{tlx} shows the reported NASA TLX scores. The height of the bar describes the mean value while the whiskers shows the SD. Numerical values are reported in Table \ref{tlx_values}.

There are no big differences between condition one and two. The only significant differences between those two conditions can be found in the \emph{performance}, t(56)=3.24, p=0.002, d=0.360 and \emph{frustration}, t(56)=2.15, p=0.036, d=0.271 metric. This means that the subjects felt less frustration and evaluated their performance as better when using the predictive display.


\begin{figure}[h!]
    \centering
    \includegraphics[scale=0.85]{nasa_tlx_bar}
    \caption{NASA TLX (task load index) results for each display type. Lower is better.}
    \label{tlx}
\end{figure}

% Please add the following required packages to your document preamble:
% \usepackage{booktabs}
\begin{table}[]
\small
\centering
\caption{Rated NASA TLX values and standard deviation (SD), N=57. Lower is better.}
\label{tlx_values}
\begin{tabularx}{\textwidth}{@{}XXXX@{}}
\toprule
Metric      & Display  & Rated value & SD \\ \midrule
Mental      & Delay    & 5.67        & 2.05               \\
            & Delay PD & 5.51        & 2.25               \\
            & No delay & 3.56        & 2.03               \\\addlinespace
Physical    & Delay    & 2.88        & 2.14               \\
            & Delay PD & 2.84        & 2.19               \\
            & No delay & 2.18        & 1.84               \\\addlinespace
Temporal    & Delay    & 5.84        & 2.08               \\
            & Delay PD & 5.67        & 2.10               \\
            & No delay & 5.39        & 2.30               \\\addlinespace
Performance & Delay    & 5.53        & 2.29               \\
            & Delay PD & 4.74        & 2.05               \\
            & No delay & 2.70        & 1.60               \\\addlinespace
Effort      & Delay    & 6.02        & 1.94               \\
            & Delay PD & 5.77        & 1.99               \\
            & No delay & 4.67        & 2.08               \\\addlinespace
Frustration & Delay    & 5.65        & 2.35               \\
            & Delay PD & 5.04        & 2.13               \\
            & No delay & 2.44        & 1.79               \\ \bottomrule
\end{tabularx}
\end{table}
\clearpage
\section{Subjective delay}

\figref{subjective_delay_norm} shows the normalized reported total delay in seconds for the three conditions. The participants reported a 11\% decrease in subjective latency using the predictive display versus the normal display with the same latency. This results is however not significant, t(56)=1.40, p=0.167, d=0.356.


\begin{figure}[h!]
    \centering
    \includegraphics[scale=0.85]{subjective_delay_norm}
    \caption{Normalized reported subjective latency in seconds.}
    \label{subjective_delay_norm}
\end{figure}

\section{Key presses}

\figref{keypresses} shows the number of key presses performed during the 90 seconds of task time for each display type. With a low latency, participants are in a greater degree trying to continuously maneuver the ROV.

\begin{figure}[h!]
    \centering
    \includegraphics[scale=0.85]{keypresses}
    \caption{The number of key presses performed during 90 seconds.}
    \label{keypresses}
\end{figure}

