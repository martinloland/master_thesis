\section{Performance}

\figref{performanceNorm} shows the normalized score, number of hits in 90 seconds, for each of the display types and all participants. A similar figure with non-normalized values can be found in the appendix at page \textbf{TODO}. \textit{Delay} refers to the \textit{added} delay, which means that "No delay" translates to the inherent system delay of $250 ms$. The numerical values together with the standard deviation (SD) is reported in table \ref{score}. In addition, the percentage difference in means between displays is also reported in that table. The statistical significance and effect size between conditions can be seen in table \ref{score2}.

Using normalized scores, participants performed on average 20.29\% better using the predictive display versus no predictive display, $t(56)=4.82$, $p<.001$, $d=0.735$. In comparison, the subjects performed on average 153.32\% better in the no delay condition versus the delayed condition without predictive help, $t(56)=23.04$, $p<.001$, $d=4.413$. All differences are statistically significant.


\begin{figure}[h!]
    \centering
    \includegraphics[scale=0.85]{performance_norm}
    \caption{Normalized score for each display type}
    \label{performanceNorm}
\end{figure}

% Please add the following required packages to your document preamble:
% \usepackage{booktabs}
\begin{table}[]
\centering
\caption{Normalized mean$\pm$SD of display scores and mean differences.}
\label{score}
\begin{tabularx}{\textwidth}{@{}lYYYY@{}}
\toprule
Display  & Score                       & \multicolumn{3}{c}{Mean difference from} \\ \cmidrule(l){3-5} 
         &                             & Delay        & Delay PD    & No delay    \\ \midrule
Delay    & 6.24 $\pm$ 1.39 & -            & -17.10\%      & -60.69\%    \\
Delay PD & 7.52 $\pm$ 1.43 & 20.62\%      & -           & -52.58\%    \\
No delay & 15.87 $\pm$ 1.99 & 154.37\%     & 110.88\%    & -           \\ \bottomrule
\end{tabularx}
\end{table}
% Please add the following required packages to your document preamble:
% \usepackage{booktabs}
\begin{table}[]
\small
\centering
\caption{Mean difference, paired samples t-test and Cohen's d effect size for display pair scores. Gamers = plays weekly or more.}
\label{score2}
\begin{tabularx}{\textwidth}{@{}lllYYYY@{}}
\toprule
\multicolumn{2}{c}{Group / Pair}    & Mean difference & \multicolumn{3}{c}{t-test for Equality of Means} & d     \\ \midrule
\multicolumn{2}{l}{}                &                 & t               & df          & p                &       \\ \cmidrule(lr){4-6}
\multicolumn{2}{l}{All N=57}        &                 &                 &             &                  &       \\
Delay             & Delay PD        & 20.62\%         & 4.80            & 56          & $<$.001          & 0.904 \\
Delay             & No delay        & 154.37\%        & 23.15           & 56          & $<$.001          & 5.569 \\
Delay PD          & No delay        & 110.88\%        & 19.66           & 56          & $<$.001          & 4.772 \\\addlinespace
\multicolumn{2}{l}{Gamers n=17}     &                 &                 &             &                  &       \\
Delay             & Delay PD        & 30.13\%         & 4.34            & 16          & $<$.001          & 1.376 \\
Delay             & No delay        & 174.64\%        & 14.93           & 16          & $<$.001          & 6.463 \\
Delay PD          & No delay        & 111.05\%        & 10.83           & 16          & $<$.001          & 4.965 \\\addlinespace
\multicolumn{2}{l}{Non-gamers n=40} &                 &                 &             &                  &       \\
Delay             & Delay PD        & 16.91\%         & 3.20            & 39          & .003          & 0.731 \\
Delay             & No delay        & 146.46\%        & 18.16           & 39          & $<$.001          & 5.237 \\
Delay PD          & No delay        & 110.80\%        & 16.21           & 39          & $<$.001          & 4.655 \\ \bottomrule
\end{tabularx}
\end{table}


\section{Task load index}

\figref{tlx} shows the reported NASA TLX scores. The height of the bar describes the mean value while the whiskers shows the SD. There are no big differences between condition one and two. The only significant differences between those two conditions can be found in the \emph{performance}, t(56)=3.24, p=0.002, d=0.360 and \emph{frustration}, t(56)=2.15, p=0.036, d=0.271 metric. This means that the subjects felt less frustration and evaluated their performance as better when using the predictive display.


\begin{figure}[h!]
    \centering
    \includegraphics[scale=0.75]{nasa_tlx_bar}
    \caption{NASA TLX (task load index) results for each display type. Lower is better.}
    \label{tlx}
\end{figure}

\section{Subjective delay}

\figref{subjective_delay} shows the reported total delay in seconds for the three conditions in absolute and normalized values. \figref{delay_abs} with absolute values describes the numerical values that the subjects reported. \figref{delay_norm} is better suited to describe the relative differences in reported delay between the conditions. Using normalized values, the participants reported a 20\% decrease in subjective latency using the predictive display versus the normal display with the same latency, t(56)=2.71, p=0.009, d=0.687.

\begin{figure}[ht!]
\begin{subfigure}{0.5\textwidth}
    \centering
    \includegraphics[width=\linewidth]{subjective_delay_abs}
    \caption{Absolute values.}
    \label{delay_abs}
\end{subfigure}
\hfill
\begin{subfigure}{0.5\textwidth}
    \centering
    \includegraphics[width=\linewidth]{subjective_delay_norm}
    \caption{Normalized values}
    \label{delay_norm}
\end{subfigure}
\caption{Reported subjective latency in seconds.}
\label{subjective_delay}
\end{figure}

\section{Key presses}

\figref{keypresses} shows the number of key presses performed during the 90 seconds of task time for each display type. With a low latency, participants are in a greater degree trying to continuously maneuver the ROV.

\begin{figure}[h!]
    \centering
    \includegraphics[scale=0.85]{keypresses}
    \caption{The number of key presses performed during 90 seconds.}
    \label{keypresses}
\end{figure}

